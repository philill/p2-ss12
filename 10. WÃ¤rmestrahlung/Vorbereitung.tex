\documentclass[ngerman, parskip*]{scrartcl}
\usepackage[utf8]{inputenc} 
\usepackage[ngerman]{babel} % deutsche Sprache

\usepackage[decimalsymbol=comma,
            loctolang={DE:ngerman,UK:english},
            separate-uncertainty = true,
            multi-part-units=single
            ]{siunitx}
\usepackage{paralist}
\usepackage{amsmath}
\usepackage{graphicx}
\usepackage{booktabs}
\usepackage{float}
\usepackage{caption}
\usepackage{subcaption}
\usepackage{tabularx}
\usepackage{array}
\usepackage{commath}
\usepackage{amsfonts}


\title{Praktikum Klassische Physik Teil 2 (P2)}
\subtitle{Vorbereitung: Wärmestrahlung}
\author{Simon Fromme, Philipp Laur}

\date{\today}

\begin{document}

\maketitle
\tableofcontents
\newpage


\section{Planck'sches Strahlungsgesetz}

Sei $n_0$ die Teilchenzahldichte der unangeregten und $n^{*}$ die der angeregten Atome eine Schwarzkörpers. Im thermischen Gleichgewicht ist das Verhältnis der beiden Dichten durch die Boltzmann-Verteilung gegeben.
\begin{align}\label{eq:boltzmann-verteilung}
  \dfrac{n^{*}}{n_0} = e^{-hf/(kT)}
\end{align}
Teilchen haben nun verschiedene Möglichkeiten zur Wechselwirkung mit der Strahlung. Sei nun nachfolgend mit $\alpha$ die Zahl der Ereignisse pro Volumen und Zeiteinheit bezeichnet. Dann gilt
\begin{enumerate}
\item Absorbtion von Strahlung der Energie $hf$ 
  \[ \alpha_{\textrm{Absorbtion}} = B \cdot \varrho(f,T) \cdot n_0 \]
\item spontane Emission von Strahlung der Energie $hf$ 
  \[ \alpha_{\textrm{Emission}} = A \cdot n^{*} \]
\item stimulierte Emission der Energie $hf$ (direkte Umkehrung der Absorbtion)
  \[ \alpha_{\textrm{st. Emission}} = B \cdot \varrho(f,T) \cdot n^{*}\]
\end{enumerate}

Im Gleichgewicht muss nun die Rate der Absorbtion gleich der Rate der Emission sein, es gilt also 
\begin{align*}
  \alpha_{\textrm{Absorbtion}} = \alpha_{\textrm{Emission}} + \alpha_{\textrm{st. Emission}}
\end{align*}
bzw.
\begin{align*}
  & B \cdot \varrho(f,T) n_0 = A \cdot n^{*} + B \cdot \varrho(f,T) n^{*} \\
  \Rightarrow & \dfrac{n^{*}}{n_0} = \dfrac{B \cdot \varrho(f,T)}{A + B \cdot \varrho(f,T)}
\end{align*}
Und mit (\ref{eq:boltzmann-verteilung}),
\begin{align*}
  & e^{-hf/(kT)} = \dfrac{B \cdot \varrho(f,T)}{A + B \cdot \varrho(f,T)} \\
  \Rightarrow & \varrho(f,T) = \dfrac{A}{B} \cdot \dfrac{e^{-hf/(kT)}}{1- e^{-hf/(kT)}} = \dfrac{A}{B} \cdot \dfrac{1}{e^{hf/(kT)}- 1}
\end{align*}
Nach weiterer Betrachtung (Eigenschwingungsspektrum Hohlraum) ergibt sich die Konstante zu
\begin{align*}
  \frac{A}{B} = \dfrac{8\pi h f^3}{c^3}
\end{align*}
und damit das Plancksche Strahlungsgesetz zu 
\begin{align*}
   \varrho(f,T) \textrm{d}f = \dfrac{8\pi h f^3}{c^3}  \cdot \dfrac{1}{e^{hf/(kT)}- 1}.
\end{align*}


\end{document}
