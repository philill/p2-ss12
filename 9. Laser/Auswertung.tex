\documentclass[ngerman, parskip*]{scrartcl}
\usepackage[utf8]{inputenc} 
\usepackage[ngerman]{babel} % deutsche Sprache

\usepackage[decimalsymbol=comma,
            loctolang={DE:ngerman,UK:english},
            separate-uncertainty = true,
            multi-part-units=single
            ]{siunitx}
\usepackage{paralist}
\usepackage{amsmath}
\usepackage{graphicx}
\usepackage{booktabs}
\usepackage{float}
\usepackage{caption}
\usepackage{subcaption}
\usepackage{tabularx}
\usepackage{array}
\usepackage{commath}
\usepackage{amsfonts}


\title{Praktikum Klassische Physik Teil 2 (P2)}
\subtitle{Laser A}
\author{Simon Fromme, Philipp Laur}

\date{\today}

\begin{document}

\maketitle
\tableofcontents
\newpage

\section{Bresterwinkel}

\subsection{Brechungsindex von Glas}

Der Bresterwinkel wird über die Reflexion des Laserstrahls an der Zimmerdecke beobachtet, für 
\begin{align*}
  \alpha_B &=  \SI{180}{\degree} - \SI{125}{\degree} \\
  &=  \SI{55}{\degree}
\end{align*} 
kann man hier ein Minimum der Intensität beobachten. Aus der bekannten Beziehung aus der Vorbereitung (Annahme für den Brechungsindex von Luft $n_{\textrm{Luft}} = 1$) ergibt sich der Brechungsindex des Glases zu
\begin{align*}
  n_{\textrm{Glas}} &= \tan(\alpha_B) \\
  &= 1,428.
\end{align*}

Dies liegt in der Größenordnung gängiger optischer Gläser.

Die Bestimmung des Brewsterwinkels über das Maximum der Transmission mithilfe eines Si-Photoelements ist ungenau, da das Spannungssignal des Photoelements während des Messvorgangs zu stark schwankt, um genau messen zu könne, wann das Maximum der Transmission erreicht ist.


\section{Beugung an Spalt, Steg, Kreisloch, Kreisblende und Kante}

\subsection{Vergleich der Beugungsfiguren von Spalt und Steg}

%% Hier fehlen die Bilder 

\subsection{Vergleich der Beugungsfiguren von Kreisöffnung, Kreisscheibe und Kante }

Im Versuch konnten die typischen (kreisförmigen) Beugungsminima bei der Beugung an Kreisöffnung und -scheibe nur sehr schlecht beobachet werden. Es konnten jedoch keine nennenswerten Unterschiede zwischen den beiden Beugungsbildern beobachtet werden, so dass man das Babinet-Theorem in diesem Fall bestätigen kann. 

%% Zuordnung der entsprechenden Beugungsbildern schwierig

\subsection{Bestimmung der Breite eines Spalts}

In diesem Versuchsteil wird die Breite eines Einfach-Spalts aus dem -nach Bestrahlung durch Laser-Licht erzeugtem- Beugungsmuster bestimmt. 

Hierzu wurde jeweils der Abstand des n-ten Beugungsminimums von der optischen Achse aus $y_n = \frac{1}{2}\cdot (y_n^{(l)} - y_n^{(r)})$ bestimmt, wobei $y_n^{(l)}$ die y-Position des links und $y_n^{(r)}$ die Position des rechts vom Hauptmaximum liegend Beugungsminimum n-ter Ordnung ist. Die ermittelten Werte sind in Tabelle \ref{table:2_1} dargestellt.

%% @TODO: Wertetabelle, Regression und Berechnung der Spaltbreite

Aus der Position des Spalts $x_1 = \SI{59,3}{\cm}$ und der Position des Schirms $x_2 = \SI{289,8}{\cm}$ ergibt sich der Abstand Spalt-Schirm zu $l = \SI{230,5}{\cm}$. Die Wellenlänge des Lasers war mit $\lambda = \SI{632,8}{\nano\meter}$ angegeben. 


\subsection{Durchmesser eines Haares aus Beugungsbild}

Nach dem Babinet-Theorem besitzt das Haar (Steg-förmig) das gleiche Beugungsbild, wie ein Spalt der selben Breite. Folglich kann die Haardicke analog zur ersten Teilaufgabe aus den Abständen der Beugungsminima bestimmt werden. Mit der bekannten Beziehung
\begin{align*}
  d = \dfrac{n\lambda l}{y_n} \Rightarrow y_n = \dfrac{\lambda \cdot l}{d}\cdot n
\end{align*}
bestimmt werden, wobei $y_n$ der Abstand des n-ten Beugungsminimums von der optischen Achse ist. 

Im Versuch wurde die Lage der Beugungsminima der Ordnung $n=1\dots 4$ vermessen. Die Messwerte sind in Tabelle \ref{table:2_4} dargestellt. 

\begin{table}[!h]
\centering
\caption{Position der Beugungsminima}
\begin{tabular}{l|cccc}
\toprule
$n$ & 1 & 2 & 3 & 4 \\ 
$y_n$ in $\si{\cm}$ & 1.90 & 3.75 & 5.65 & 7.45 \\ 
\bottomrule
\end{tabular}
\label{table:2_4}
\end{table}

Wie in 2.1 wird nun $y_n$ gegen $n$ aufgetragen und aus der Steigung der Regressionsgerade $m$ die Dicke des Haars zu $d = \frac{\lambda l}{m}$ bestimmt.

\begin{figure}
  \centering
  \caption{Regression zur Bestimmung von $m = \dfrac{\lambda \cdot l}{d}$ }
  \input{Diagramme/2_4.tex}
\end{figure}

Es ergibt sich 
\begin{align*}
  m = \SI {1,855}{\cm}
\end{align*}
und mit der Wellenlänge des Lasers $\lambda = \SI{632,8}{\nano\meter}$ und $l = \SI{289,8}{\cm} - \SI{56,2}{\cm} = \SI{233,6}{\cm}$ die Dicke des Haares schließlich zu
\begin{align*}
  d = \SI{79,69}{\micro\meter}.
\end{align*}
Mit der Micrometerschraube wurde eine Dicke von $d = \SI{55}{\micro\meter}$ bestimmt, was deutlich unter dem Wert aus dem Beugungsexperiment liegt. Dies kann dadurch erklärt werden, dass das Haar hier möglicherweise gequetscht wurde. 


%% Beginn: Aufgabenteil Philipp


\section{Beugung an Mehrfachspalt und Gittern}

\subsection{Beugungsbilder von Doppel- und Dreifachspalt}

\begin{figure}
\centering
        \begin{subfigure}[!h]{0.49\textwidth}
          \centering
          \includegraphics[width=\textwidth,natwidth=2560,natheight=1920]{Bilder/Doppelspalt_1.jpg}
          \caption{Doppelspalt, $d = \SI{0,25}{\mm}$, $b = \SI{0,50}{\mm}$}
        \end{subfigure}
        \begin{subfigure}[!h]{0.49\textwidth}
          \centering
          \includegraphics[width=\textwidth,natwidth=2560,natheight=1920]{Bilder/Doppelspalt_2.jpg}
          \caption{Doppelspalt, $d = \SI{0,25}{\mm}$, $b = \SI{0,75}{\mm}$}
        \end{subfigure}
\end{figure}

\begin{figure}
\centering
        \begin{subfigure}[!h]{0.49\textwidth}
          \centering
          \includegraphics[width=\textwidth,natwidth=2560,natheight=1920]{Bilder/Doppelspalt_1.jpg}
          \caption{Doppelspalt, $d = \SI{0,25}{\mm}$, $b = \SI{0,50}{\mm}$}
        \end{subfigure}
        \begin{subfigure}[!h]{0.49\textwidth}
          \centering
          \includegraphics[width=\textwidth,natwidth=2560,natheight=1920]{Bilder/Dreifachspalt.jpg}
          \caption{Dreifachspalt}
        \end{subfigure}
\end{figure}


\subsection{Beugungsbilder von Kreuz- und Wabengittern}

\begin{figure}
\centering
        \begin{subfigure}[!h]{0.49\textwidth}
          \centering
          \includegraphics[width=\textwidth,natwidth=2560,natheight=1920]{Bilder/Kreuzgitter.jpg}
          \caption{Kreuzgitter}
        \end{subfigure}
        \begin{subfigure}[!h]{0.49\textwidth}
          \centering
          \includegraphics[width=\textwidth,natwidth=2560,natheight=1920]{Bilder/Wabengitter.jpg}
          \caption{Wabengitter}
        \end{subfigure}
\end{figure}





\end{document}
