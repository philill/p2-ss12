\documentclass[ngerman]{scrartcl}
\usepackage[latin1]{inputenc}% erm\"oglich die direkte Eingabe der Umlaute 
\usepackage[T1]{fontenc} % das Trennen der Umlaute
\usepackage{ngerman} % hiermit werden deutsche Bezeichnungen genutzt und 
                     % die W\"orter werden anhand der neue Rechtschreibung 
		     % automatisch getrennt.
\usepackage[decimalsymbol=comma,
            loctolang={DE:ngerman,UK:english},
            separate-uncertainty = true,
            multi-part-units=single
            ]{siunitx}
\usepackage{paralist}
\usepackage{amsmath}
\usepackage{graphicx}
\usepackage{booktabs}
\usepackage{float}
\usepackage{caption}
\usepackage{subcaption}
\usepackage{tabularx}
\usepackage{array}
\usepackage{commath}
\usepackage{amsfonts}


\title{Praktikum Klassische Physik Teil 2 (P2)}
\subtitle{Interferenz}
\author{Simon Fromme, Philipp Laur}

\date{\today}

\begin{document}

\parindent 0pt

\maketitle
\tableofcontents
\newpage

\section{Beugung am Gitter}
\label{sec:beugung-am-gitter}

\subsection{Justierung des Gitterspektrometer}
\label{sec:just-des-gitt}

Alle Vorgaben zur Justierung und Scharfstellung des Gitterspektrometers werden der Aufgabenstellung entsprechend durchgef�hrt. Beachtenswert ist, dass die Scharfstellung durch Philipp Laur (normalsichtig), einige Ablesungen jedoch von Simon Fromme (leicht kurzsichtig) durchgef�hrt wurden. 

\subsection{Bestimmung der Gitterkonstante eines Gitters}
\label{sec:best-der-gitt}









\end{document}