\documentclass[ngerman]{scrartcl}
\usepackage[latin1]{inputenc}% erm\"oglich die direkte Eingabe der Umlaute 
\usepackage[T1]{fontenc} % das Trennen der Umlaute
\usepackage{ngerman} % hiermit werden deutsche Bezeichnungen genutzt und 
                     % die W\"orter werden anhand der neue Rechtschreibung 
		     % automatisch getrennt.
\usepackage[decimalsymbol=comma,
            loctolang={DE:ngerman,UK:english},
            separate-uncertainty = true,
            multi-part-units=single
            ]{siunitx}
\usepackage{paralist}
\usepackage{amsmath}
\usepackage{graphicx}
\usepackage{booktabs}
\usepackage{float}
\usepackage{caption}
\usepackage{subcaption}
\usepackage{tabularx}
\usepackage{array}
\usepackage{commath}
\usepackage{amsfonts}


\title{Praktikum Klassische Physik Teil 2 (P2)}
\subtitle{Franck-Hertz Versuch}
\author{Simon Fromme, Philipp Laur}

\date{\today}

\begin{document}

\parindent 0pt

\maketitle
\tableofcontents
\newpage

\section{Quecksilber-Franck-Hertz-R�hre}
\label{sec:quecks-franck-hertz}



\subsection{Aufbau der Schaltung}
\label{sec:aufbau-der-schaltung}

Die Schaltung wurde wie in der Versuchsbeschreibung angegeben aufgebaut. Die Funktionsweise wurde bereits in der Vorbereitung diskutiert.

\subsection{Anregungsenergie und Kontaktspannung}
\label{sec:anreg-und-kont}

% Einfluss von Kathodenheizung, Spannung am Raumladungsgitter, Gegenspannung auf die Form der Franck Hertz-Kurve (qualitativ)

Mit Hilfe eines Oszilloskops werden zun�chst die Franck Hertz Kurve f�r die Temperaturen $T= \SI{170}{\celsius}, \SI{160}{\celsius}, \SI{150}{\celsius}, \SI{140}{\celsius}, \SI{120}{\celsius}$ aufgenommen. 

Die Betriebsparameter werden jeweils so eingestellt, dass die charakteristische Form der Franck-Hertz-Kurve gut sichtbar ist. Gemessen wird jeweils die Auff�gerspannung $U_A$ in Abh�ngigkeit der Beschleunigungsspannung $U_1$. 

\begin{table}[htbp]
\centering
\caption{}
\begin{tabular}{rrrrr}
\toprule
T in $\si{\celsius}$ & $U_1$ in $\si{\volt}$ & $U_2$ in $\si{\volt}$ & $U_3$ in $\si{\volt}$ & Kathodenspannung in $\si{\volt}$ \\ 
\midrule
170 & 5,26 & 19,00 & 0,51 & 6,4 \\ 
160 & 4,75 & 19,00 & 0,45 & 6,4 \\ 
150 & 3,70 & 19,00 & 0,95 & 6,4 \\ 
140 & 2,44 & 19,00 & 0,71 & 6,4 \\ 
120 & 5,01 & 19,00 & 2,54 & 4,9 \\ 
\bottomrule
\end{tabular}
\label{}
\end{table}

% Diskussion des Zustandekommens der typischen Franck-Hertz Kurve



% Berechnung:
% Anregungsenergie (niedrigste Anregung)

Wie in der Vorbereitung gezeigt tritt bei den vorliegenden Versuchsparametern nur eine Anregung in den ersten angeregten Zustand auf, so dass der Abstands zweier Peaks der Beschleunigungsspannung die \textbf{Anregungsenergie} (in $\si{\eV}$) angibt.  Mittelt man nun bei verschiedenen Temperaturen �ber alles Peakabst�nde, so erh�lt man  
\begin{align*}
  E = e\cdot \bar{\Delta U} = \SI{5,28}{\eV}.
\end{align*}

% Kontaktspannung zwischen Kathode und Anode
Die \textbf{Kontaktspannung} l�sst sich mit der Beziehung
\begin{align*}
  U_K = n\cdot \overline{\Delta U} - U_1 - U_{2,n} 
\end{align*}
berechnen, wobei $\overline{\Delta U}$ die mittlere Spannung zwischen zwei Peaks, $U_{2,n}$ die Beschleunigungsspannung am $n$-ten Peak und $U_1$ die Spannung am Hilfsgitter ist. Beachtet werden musste, dass der erste Peak au�er bei $T=\SI{120}{\celsius}$ nicht beobachtet werden konnte, so dass dort der erste gemessene dem zweiten auftretenden Peak entspricht. 

Es ergeben sich folgende Thermokontaktspannungen f�r die unterschiedlichen Temperaturen:
\begin{table}[htbp]
\centering
\caption{Thermokontaktspannungen bei verschiedenen Temperaturen}
\begin{tabular}{rrrrr}
\toprule
T in $\si{\celsius}$ & $U_\mathrm{Th.}$  \\ 
\midrule
170 & -2,15 \\ 
160 & -1,41 \\ 
150 & -1,91 \\ 
140 & -1,27 \\ 
120 & -3,08 \\  
\bottomrule
\end{tabular}
\label{}
\end{table}

Da die Thermokotaktspannung temperaturabh�ngig ist, wurde auf auf die Bildung des Mittelwertes verzichtet und nachfolgende jeweils um die Thermokontaktspannung der entsprechenden Temperatur korrigiert. 


\subsection{Anodenstromkurve}
\label{sec:anodenstromkurve}

F�r die Aufnahme der Anodenstromkurve wird die Temperatur gem�� Aufgabenstellung auf $T = \SI{150}{\celsius}$ eingestellt. Nun wird der Anodenstrom $I_A$ in Abh�ngigkeit der Beschleunigungsspannung $U_2$ gemessen. 

Die Messwerte sind in Tabelle \ref{table:anodenstromkurve} angegeben. 

\begin{table}[htbp]
\centering
\caption{Messerte: Bestimmung der Anodenstromkurve}
\begin{tabular}{rr|rr}
\toprule
U in V & I in $\si{\micro\ampere}$ & U in V & I in $\si{\micro\ampere}$ \\ 
\midrule
0,0 & 0,00 & 16,0 & 0,41 \\ 
2,1 & 0,05 & 18,1 & 0,47 \\ 
4,1 & 0,10 & 19,9 & 0,54 \\ 
6,0 & 0,14 & 21,9 & 0,66 \\ 
7,9 & 0,18 & 24,1 & 0,80 \\ 
10,2 & 0,25 & 26,2 & 0,94 \\ 
12,1 & 0,30 & 27,5 & 1,07 \\ 
14,0 & 0,35 & 30,4 & 1,44 \\ 
\bottomrule
\end{tabular}
\label{table:anodenstromkurve}
\end{table}

% Anodenstrom - Thermokontaktspannung
\begin{figure}
  \centering
  \caption{Anodenstromkurve bei $T = \SI{150}{\celsius}$}
  % GNUPLOT: LaTeX picture
\setlength{\unitlength}{0.240900pt}
\ifx\plotpoint\undefined\newsavebox{\plotpoint}\fi
\sbox{\plotpoint}{\rule[-0.200pt]{0.400pt}{0.400pt}}%
\begin{picture}(1500,900)(0,0)
\sbox{\plotpoint}{\rule[-0.200pt]{0.400pt}{0.400pt}}%
\put(171.0,131.0){\rule[-0.200pt]{4.818pt}{0.400pt}}
\put(151,131){\makebox(0,0)[r]{ 0}}
\put(1419.0,131.0){\rule[-0.200pt]{4.818pt}{0.400pt}}
\put(171.0,313.0){\rule[-0.200pt]{4.818pt}{0.400pt}}
\put(151,313){\makebox(0,0)[r]{ 0.5}}
\put(1419.0,313.0){\rule[-0.200pt]{4.818pt}{0.400pt}}
\put(171.0,495.0){\rule[-0.200pt]{4.818pt}{0.400pt}}
\put(151,495){\makebox(0,0)[r]{ 1}}
\put(1419.0,495.0){\rule[-0.200pt]{4.818pt}{0.400pt}}
\put(171.0,677.0){\rule[-0.200pt]{4.818pt}{0.400pt}}
\put(151,677){\makebox(0,0)[r]{ 1.5}}
\put(1419.0,677.0){\rule[-0.200pt]{4.818pt}{0.400pt}}
\put(171.0,859.0){\rule[-0.200pt]{4.818pt}{0.400pt}}
\put(151,859){\makebox(0,0)[r]{ 2}}
\put(1419.0,859.0){\rule[-0.200pt]{4.818pt}{0.400pt}}
\put(171.0,131.0){\rule[-0.200pt]{0.400pt}{4.818pt}}
\put(171,90){\makebox(0,0){ 0}}
\put(171.0,839.0){\rule[-0.200pt]{0.400pt}{4.818pt}}
\put(357.0,131.0){\rule[-0.200pt]{0.400pt}{4.818pt}}
\put(357,90){\makebox(0,0){ 5}}
\put(357.0,839.0){\rule[-0.200pt]{0.400pt}{4.818pt}}
\put(544.0,131.0){\rule[-0.200pt]{0.400pt}{4.818pt}}
\put(544,90){\makebox(0,0){ 10}}
\put(544.0,839.0){\rule[-0.200pt]{0.400pt}{4.818pt}}
\put(730.0,131.0){\rule[-0.200pt]{0.400pt}{4.818pt}}
\put(730,90){\makebox(0,0){ 15}}
\put(730.0,839.0){\rule[-0.200pt]{0.400pt}{4.818pt}}
\put(917.0,131.0){\rule[-0.200pt]{0.400pt}{4.818pt}}
\put(917,90){\makebox(0,0){ 20}}
\put(917.0,839.0){\rule[-0.200pt]{0.400pt}{4.818pt}}
\put(1103.0,131.0){\rule[-0.200pt]{0.400pt}{4.818pt}}
\put(1103,90){\makebox(0,0){ 25}}
\put(1103.0,839.0){\rule[-0.200pt]{0.400pt}{4.818pt}}
\put(1290.0,131.0){\rule[-0.200pt]{0.400pt}{4.818pt}}
\put(1290,90){\makebox(0,0){ 30}}
\put(1290.0,839.0){\rule[-0.200pt]{0.400pt}{4.818pt}}
\put(171.0,131.0){\rule[-0.200pt]{0.400pt}{175.375pt}}
\put(171.0,131.0){\rule[-0.200pt]{305.461pt}{0.400pt}}
\put(1439.0,131.0){\rule[-0.200pt]{0.400pt}{175.375pt}}
\put(171.0,859.0){\rule[-0.200pt]{305.461pt}{0.400pt}}
\put(30,495){\makebox(0,0){\rotatebox{90}{$I$ in $\si{\micro\ampere}$}}}
\put(805,29){\makebox(0,0){$U_2$ in $\si{\V}$}}
\put(171,131){\makebox(0,0){$+$}}
\put(249,149){\makebox(0,0){$+$}}
\put(324,167){\makebox(0,0){$+$}}
\put(395,182){\makebox(0,0){$+$}}
\put(466,197){\makebox(0,0){$+$}}
\put(551,222){\makebox(0,0){$+$}}
\put(622,240){\makebox(0,0){$+$}}
\put(693,258){\makebox(0,0){$+$}}
\put(768,280){\makebox(0,0){$+$}}
\put(846,302){\makebox(0,0){$+$}}
\put(913,328){\makebox(0,0){$+$}}
\put(988,371){\makebox(0,0){$+$}}
\put(1070,422){\makebox(0,0){$+$}}
\put(1148,473){\makebox(0,0){$+$}}
\put(1197,520){\makebox(0,0){$+$}}
\put(1305,655){\makebox(0,0){$+$}}
\put(171.0,131.0){\rule[-0.200pt]{0.400pt}{175.375pt}}
\put(171.0,131.0){\rule[-0.200pt]{305.461pt}{0.400pt}}
\put(1439.0,131.0){\rule[-0.200pt]{0.400pt}{175.375pt}}
\put(171.0,859.0){\rule[-0.200pt]{305.461pt}{0.400pt}}
\end{picture}

\end{figure}

Zur �berpr�fung der Beziehung
\begin{align*}
  I_A = \lambda U_2^{\dfrac{3}{2}}
\end{align*}
wird $\ln(I_A$ gegen $\ln(U_2$ abgetragen. Legt man nun eine Regressionsgerade durch diese Punkte, so entspricht deren Steigung gerade dem gesuchten Exponenten. Der ermittelte Wert
\begin{align*}
  m = 1,20228
\end{align*}
weicht jedoch relativ stark von $m = 1,5$ ab und der relative Fehler betr�gt 19,8\%.
 
\begin{figure}
  \centering
  \caption{logarithmische Anodenstromkuve zur Bestimmung von $m$}
  \input{Diagramme/anodenstromkurve-logarithmisch.tex}
\end{figure}


\subsection{Ionisierungsarbeit von Quecksilber}
\label{sec:ionis-von-quecks}

In diesem Versuchsteil wird das Gitter $G_1$ als Beschleunigungsgitter genutzt. Hierbei liegt $G_1$ auf dem selben Potential wie $G_2$. Als Temperatur wird $T = \SI{120}{\celsius}$ gew�hlt, wobei die Betriebsparameter entsprechend 1.2 eingestellt werden. 

Zun�chst wird der Anodenstrom gegen die Beschleunigungsspannung (korrigiert um Thermokontaktspannung) aufgetragen.

Im Diagramm erkennt man einen typischen Knick, dessen Spannungswert der Ionisierungsarbeit in $\si{\eV}$ entspricht. Legt man durch die Punkte vor und nach dem Knick eine Regressionsgerade und bestimmt deren Schnittpunkt, so erh�lt man
\begin{align*}
  E_{\mathrm{Ion.}} \approx \SI{17}{\eV}
\end{align*}
Verglichen mit dem Literaturwert von $E_{\mathrm{Ion.}} = \SI{10,44}{\eV}$ ergibt sich jedoch eine zu gro�e Abweichung von, so dass ein Fehler in der Rechnung aufgetreten sein muss.

Nun wird der Auff�ngerstrom gegen die Beschleunigungsspannung aufgetragen und mittels Oszilloskop (\textit{Picoscope}) die Position der Spannungsspitze ermittelt. Diese liegt bei $U_2 = \SI{13,38}{\volt}$. Korrigiert man um die Thermokontaktspannung, so erh�lt man $U = \SI{11,11}{\volt}$, bzw. 
\begin{align*}
  E_{\mathrm{Ion.}} = \SI{11,11}{\eV}
\end{align*}

\section{Bestimmung der Energie f�r die n�chsth�here Anregung von Quecksilber}
Zur Bestimmung der h�heren Anregungsniveaus $a_i$ des Quecksilbers wurde der gleiche Aufbau wie 1.4 gew�hlt, um wieder den langen Sto�raum zu haben. Die Messungen wurden bei $T=\SI{150}{\celsius}$ durchgef�hrt, weshalb die Beschleinigungsspannung um die entsprechende Thermokontaktspannung korriegiert werden muss. Insgesamt k�nnten ??
Peaks erkannt werden. Diese lassen sich als Linearkombination der ersten beiden Anregungszust�nde $a_1=\SI{4,9}{\volt}$ und $a_1=\SI{6,7}{\volt}$ darstellen: $p_i=\lambda_i\cdot a_1+\mu_i\cdot a_2$.\\
Die Werte f�r $\lambda$ und $\mu$ wurden extrapoliert.
\renewcommand{\arraystretch}{1.3}
\begin{table}[htbp]
\centering
\caption{Anregungszust�nde}
\begin{tabular}{rrrrrrr}
\toprule
i & $U_{unkorr}$ & $U_{korr}$ (V) & $\lambda_i$ & $\mu_i$ & Energie (eV) & rel. Abweichung \\
\midrule
1 & 6,8 & 4,9 & 1 & 0 & 4,9 & 0 \\
2 & 11,9 & 10 & 2 & 0 & 9,8 & -0,02 \\
3 & 13,3 & 11,4 & 1 & 1 & 11,6 & 0,02 \\
4 & 17,6 & 15,7 & 2 & 1 & 16,5 & 0,05 \\
5 & 19,5 & 17,6 & 1 & 2 & 18,3 & 0,04 \\
6 & 21,7 & 19,8 & 4 & 0 & 19,6 & -0,01 \\
7 & 22,8 & 20,9 & 3 & 1 & 21,4 & 0,02 \\
8 & 24,1 & 22,2 & 2 & 2 & 23,2 & 0,04 \\
9 & 26,3 & 24,4 & 5 & 0 & 24,5 & 0 \\
10 & 28,5 & 26,6 & 4 & 1 & 26,3 & -0,01 \\
11 & 29,6 & 27,7 & 3 & 2 & 28,1 & 0,01 \\
12 & 30,6 & 28,7 & 2 & 3 & 29,9 & 0,04 \\
\bottomrule
\end{tabular}
\label{tab:addlabel}
\end{table}\\
Die Thermokontaktspannung wurde mit $U_{thermo, \SI{150}{\celsius}}=\SI{1,8}{\volt}$
angenommen (der zum Zeitpunkt bestimmte Wert macht keinen Sinn. Die Annahme wurde getroffen, sodass der erste Anregungszustand keine Abweichung hat). Die Werte passen sehr gut mit den theoretischen Werten �berein, die rel. Abweichung liegt zwischen minus zwei und plus vier Prozent. Diese Abweichungen sind erkl�rbar durch die nicht immer exakte Sichtbarkeit und damit verbundene Messungenauigkeit, au�erdem ist die Thermokontaktspannung fehlerbehaftet.\\
Dennoch ist Linearkombination aus den beiden Anfangsenergien gut erkennbar.
\section{Bestimmung der mittleren Anregungsenergie von Neon}
Es wurde eine Franck-Hertz-R�hre gef�llt mit Neon verwendet, die Temperatur bleibt die Raumtemperatur, da Neon bei hier schon gasf�rmig vorliegt.\\
Wir starteten die Beobachtung bei niedrigen Beschleunigungsspannungen. Zuerst entsteht ein Leuchtstreifen in der n�he von G2, dieser wandert mit steigender Spannung zu G1. Ist die Beschleuningungsspannung hoch genug um die Elektronen erneut auf Anregungsenergie zu beschleunigen entsteht ein weiterer Streifen, insgesamt waren 3 Streifen sichtbar.\\
Bei maximaler Beschleunigungsspannung, $U=\SI{70}{\volt}$ wurde dann die Kurve ausgemessen. Die Abst�nde Der Peaks lauten $\Delta U_1=\SI{17}{\volt}$, $\Delta U_2=\SI{19}{\volt}$ und $\Delta U_3=\SI{16}{\volt}$.\\
Daraus ergibt sich eine mittlere Anregungsenergie:
\begin{align*}
\Delta \overline{E}=\SI{17,3}{\eV}
\end{align*}
Die Messwerte �berschneiden sich mit den Literaturwerten. Die Ausmessung an dem Oszilloskop ist ungenau, daher die deutliche Abweichung.

\end{document}